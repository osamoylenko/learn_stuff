\documentclass[a4paper,12pt]{article}

\usepackage[T1,T2A]{fontenc} \usepackage[utf8x]{inputenc}
\usepackage[english,russian]{babel}


\usepackage{geometry}
\geometry{left=2cm}
\geometry{right=2cm}
\geometry{top=2cm}
\geometry{bottom=2cm}

\pretolerance10000
\setlength{\parindent}{0pt}
%\setlength{\parskip}{10pt}

\begin{document}

\section{Заявление}

В квитанциях, начиная с мая 2018, снова стала появляться услуга "техобслуживание газового оборудования" от навязанного мне поставщика ООО «ПРОЕКТ-СЕРВИС ГРУПП».
Сообщаю, что:
\begin{enumerate}
	\item я не давал и не даю согласия на принятие офферты в соответствии со ст. 435 ГК РФ на заключение договора на ТО ВКГО с ООО «ПРОЕКТ-СЕРВИС ГРУПП»;
	\item я не уполномачивал третьих лиц (в т.ч. ООО "МосОблЕИРЦ" и ОАО "Жилкомплекс") на заключение такого договора от моего имени;
	\item 19.05.2017 я уже писал заявление на имя генерального директора ОАО "Жилкомплекс" Котова П.В. на исключение оплаты за техническое обслуживание внутриквартирного газового оборудования со специализированной сторонней организацией ООО «ПРОЕКТ-СЕРВИС ГРУПП»;
	\item я оставляю за собой право заключить договор на ТО ВКГО с организацией по моему выбору в соответствии с Постановлением Правительства РФ от 06.05.2011 N 354 и Постановлением Правительства Российской Федерации от 14 мая 2013 г. N 410.
\end{enumerate}

Требую:
\begin{enumerate}
	\item прекратить начисление данной услуги;
	\item пересчитать за уже начисленные месяцы;
	\item предотвратить в дальнейшем практику навязывания услуг от ООО «ПРОЕКТ-СЕРВИС ГРУПП».
\end{enumerate}

Оставляю за собой право обращения в контролирующие и судебные органы.\\

Самойленко О.С.
11.07.2018

\end{document}
